\documentclass[12pt,]{krantz}
\usepackage{lmodern}
\usepackage{amssymb,amsmath}
\usepackage{ifxetex,ifluatex}
\usepackage{fixltx2e} % provides \textsubscript
\ifnum 0\ifxetex 1\fi\ifluatex 1\fi=0 % if pdftex
  \usepackage[T1]{fontenc}
  \usepackage[utf8]{inputenc}
\else % if luatex or xelatex
  \ifxetex
    \usepackage{mathspec}
  \else
    \usepackage{fontspec}
  \fi
  \defaultfontfeatures{Ligatures=TeX,Scale=MatchLowercase}
\fi
% use upquote if available, for straight quotes in verbatim environments
\IfFileExists{upquote.sty}{\usepackage{upquote}}{}
% use microtype if available
\IfFileExists{microtype.sty}{%
\usepackage[]{microtype}
\UseMicrotypeSet[protrusion]{basicmath} % disable protrusion for tt fonts
}{}
\PassOptionsToPackage{hyphens}{url} % url is loaded by hyperref
\usepackage[unicode=true]{hyperref}
\PassOptionsToPackage{usenames,dvipsnames}{color} % color is loaded by hyperref
\hypersetup{
            pdftitle={Álgebra Lineal},
            pdfauthor={Synergy Vision},
            colorlinks=true,
            linkcolor=Maroon,
            citecolor=Blue,
            urlcolor=Blue,
            breaklinks=true}
\urlstyle{same}  % don't use monospace font for urls
\usepackage{natbib}
\bibliographystyle{apalike}
\usepackage{color}
\usepackage{fancyvrb}
\newcommand{\VerbBar}{|}
\newcommand{\VERB}{\Verb[commandchars=\\\{\}]}
\DefineVerbatimEnvironment{Highlighting}{Verbatim}{commandchars=\\\{\}}
% Add ',fontsize=\small' for more characters per line
\usepackage{framed}
\definecolor{shadecolor}{RGB}{248,248,248}
\newenvironment{Shaded}{\begin{snugshade}}{\end{snugshade}}
\newcommand{\KeywordTok}[1]{\textcolor[rgb]{0.13,0.29,0.53}{\textbf{#1}}}
\newcommand{\DataTypeTok}[1]{\textcolor[rgb]{0.13,0.29,0.53}{#1}}
\newcommand{\DecValTok}[1]{\textcolor[rgb]{0.00,0.00,0.81}{#1}}
\newcommand{\BaseNTok}[1]{\textcolor[rgb]{0.00,0.00,0.81}{#1}}
\newcommand{\FloatTok}[1]{\textcolor[rgb]{0.00,0.00,0.81}{#1}}
\newcommand{\ConstantTok}[1]{\textcolor[rgb]{0.00,0.00,0.00}{#1}}
\newcommand{\CharTok}[1]{\textcolor[rgb]{0.31,0.60,0.02}{#1}}
\newcommand{\SpecialCharTok}[1]{\textcolor[rgb]{0.00,0.00,0.00}{#1}}
\newcommand{\StringTok}[1]{\textcolor[rgb]{0.31,0.60,0.02}{#1}}
\newcommand{\VerbatimStringTok}[1]{\textcolor[rgb]{0.31,0.60,0.02}{#1}}
\newcommand{\SpecialStringTok}[1]{\textcolor[rgb]{0.31,0.60,0.02}{#1}}
\newcommand{\ImportTok}[1]{#1}
\newcommand{\CommentTok}[1]{\textcolor[rgb]{0.56,0.35,0.01}{\textit{#1}}}
\newcommand{\DocumentationTok}[1]{\textcolor[rgb]{0.56,0.35,0.01}{\textbf{\textit{#1}}}}
\newcommand{\AnnotationTok}[1]{\textcolor[rgb]{0.56,0.35,0.01}{\textbf{\textit{#1}}}}
\newcommand{\CommentVarTok}[1]{\textcolor[rgb]{0.56,0.35,0.01}{\textbf{\textit{#1}}}}
\newcommand{\OtherTok}[1]{\textcolor[rgb]{0.56,0.35,0.01}{#1}}
\newcommand{\FunctionTok}[1]{\textcolor[rgb]{0.00,0.00,0.00}{#1}}
\newcommand{\VariableTok}[1]{\textcolor[rgb]{0.00,0.00,0.00}{#1}}
\newcommand{\ControlFlowTok}[1]{\textcolor[rgb]{0.13,0.29,0.53}{\textbf{#1}}}
\newcommand{\OperatorTok}[1]{\textcolor[rgb]{0.81,0.36,0.00}{\textbf{#1}}}
\newcommand{\BuiltInTok}[1]{#1}
\newcommand{\ExtensionTok}[1]{#1}
\newcommand{\PreprocessorTok}[1]{\textcolor[rgb]{0.56,0.35,0.01}{\textit{#1}}}
\newcommand{\AttributeTok}[1]{\textcolor[rgb]{0.77,0.63,0.00}{#1}}
\newcommand{\RegionMarkerTok}[1]{#1}
\newcommand{\InformationTok}[1]{\textcolor[rgb]{0.56,0.35,0.01}{\textbf{\textit{#1}}}}
\newcommand{\WarningTok}[1]{\textcolor[rgb]{0.56,0.35,0.01}{\textbf{\textit{#1}}}}
\newcommand{\AlertTok}[1]{\textcolor[rgb]{0.94,0.16,0.16}{#1}}
\newcommand{\ErrorTok}[1]{\textcolor[rgb]{0.64,0.00,0.00}{\textbf{#1}}}
\newcommand{\NormalTok}[1]{#1}
\usepackage{longtable,booktabs}
% Fix footnotes in tables (requires footnote package)
\IfFileExists{footnote.sty}{\usepackage{footnote}\makesavenoteenv{long table}}{}
\usepackage{graphicx,grffile}
\makeatletter
\def\maxwidth{\ifdim\Gin@nat@width>\linewidth\linewidth\else\Gin@nat@width\fi}
\def\maxheight{\ifdim\Gin@nat@height>\textheight\textheight\else\Gin@nat@height\fi}
\makeatother
% Scale images if necessary, so that they will not overflow the page
% margins by default, and it is still possible to overwrite the defaults
% using explicit options in \includegraphics[width, height, ...]{}
\setkeys{Gin}{width=\maxwidth,height=\maxheight,keepaspectratio}
\IfFileExists{parskip.sty}{%
\usepackage{parskip}
}{% else
\setlength{\parindent}{0pt}
\setlength{\parskip}{6pt plus 2pt minus 1pt}
}
\setlength{\emergencystretch}{3em}  % prevent overfull lines
\providecommand{\tightlist}{%
  \setlength{\itemsep}{0pt}\setlength{\parskip}{0pt}}
\setcounter{secnumdepth}{5}
% Redefines (sub)paragraphs to behave more like sections
\ifx\paragraph\undefined\else
\let\oldparagraph\paragraph
\renewcommand{\paragraph}[1]{\oldparagraph{#1}\mbox{}}
\fi
\ifx\subparagraph\undefined\else
\let\oldsubparagraph\subparagraph
\renewcommand{\subparagraph}[1]{\oldsubparagraph{#1}\mbox{}}
\fi

% set default figure placement to htbp
\makeatletter
\def\fps@figure{htbp}
\makeatother

\usepackage[T1]{fontenc}
\usepackage[utf8]{inputenc} % recommended encoding
\usepackage[spanish]{babel}
\usepackage{booktabs}
\usepackage{longtable}
\usepackage[bf,singlelinecheck=off]{caption}

%\setmainfont[UprightFeatures={SmallCapsFont=AlegreyaSC-Regular}]{Alegreya}

\usepackage{framed,color}
\definecolor{shadecolor}{RGB}{248,248,248}

\renewcommand{\textfraction}{0.05}
\renewcommand{\topfraction}{0.8}
\renewcommand{\bottomfraction}{0.8}
\renewcommand{\floatpagefraction}{0.75}

\renewenvironment{quote}{\begin{VF}}{\end{VF}}
\let\oldhref\href
\renewcommand{\href}[2]{#2\footnote{\url{#1}}}

\ifxetex
  \usepackage{letltxmacro}
  \setlength{\XeTeXLinkMargin}{1pt}
  \LetLtxMacro\SavedIncludeGraphics\includegraphics
  \def\includegraphics#1#{% #1 catches optional stuff (star/opt. arg.)
    \IncludeGraphicsAux{#1}%
  }%
  \newcommand*{\IncludeGraphicsAux}[2]{%
    \XeTeXLinkBox{%
      \SavedIncludeGraphics#1{#2}%
    }%
  }%
\fi

\makeatletter
\newenvironment{kframe}{%
\medskip{}
\setlength{\fboxsep}{.8em}
 \def\at@end@of@kframe{}%
 \ifinner\ifhmode%
  \def\at@end@of@kframe{\end{minipage}}%
  \begin{minipage}{\columnwidth}%
 \fi\fi%
 \def\FrameCommand##1{\hskip\@totalleftmargin \hskip-\fboxsep
 \colorbox{shadecolor}{##1}\hskip-\fboxsep
     % There is no \\@totalrightmargin, so:
     \hskip-\linewidth \hskip-\@totalleftmargin \hskip\columnwidth}%
 \MakeFramed {\advance\hsize-\width
   \@totalleftmargin\z@ \linewidth\hsize
   \@setminipage}}%
 {\par\unskip\endMakeFramed%
 \at@end@of@kframe}
\makeatother

\renewenvironment{Shaded}{\begin{kframe}}{\end{kframe}}

\newenvironment{rmdblock}[1]
  {
  \begin{itemize}
  \renewcommand{\labelitemi}{
    \raisebox{-.7\height}[0pt][0pt]{
      {\setkeys{Gin}{width=3em,keepaspectratio}\includegraphics{images/#1}}
    }
  }
  \setlength{\fboxsep}{1em}
  \begin{kframe}
  \item
  }
  {
  \end{kframe}
  \end{itemize}
  }
\newenvironment{rmdnote}
  {\begin{rmdblock}{note}}
  {\end{rmdblock}}
\newenvironment{rmdcaution}
  {\begin{rmdblock}{caution}}
  {\end{rmdblock}}
\newenvironment{rmdimportant}
  {\begin{rmdblock}{important}}
  {\end{rmdblock}}
\newenvironment{rmdtip}
  {\begin{rmdblock}{tip}}
  {\end{rmdblock}}
\newenvironment{rmdwarning}
  {\begin{rmdblock}{warning}}
  {\end{rmdblock}}

\usepackage{makeidx}
\makeindex

\urlstyle{tt}

\usepackage{amsthm}
\makeatletter
\def\thm@space@setup{%
  \thm@preskip=8pt plus 2pt minus 4pt
  \thm@postskip=\thm@preskip
}
\makeatother

\frontmatter

\title{Álgebra Lineal}
\providecommand{\subtitle}[1]{}
\subtitle{Ciencia de los Datos Financieros}
\author{Synergy Vision}
\date{2018-05-24}

\usepackage{amsthm}
\newtheorem{theorem}{Teorema}[chapter]
\newtheorem{lemma}{Lema}[chapter]
\theoremstyle{definition}
\newtheorem{definition}{Definición}[chapter]
\newtheorem{corollary}{Corolario}[chapter]
\newtheorem{proposition}{Proposición}[chapter]
\theoremstyle{definition}
\newtheorem{example}{Ejemplo}[chapter]
\theoremstyle{definition}
\newtheorem{exercise}{Ejercicio}[chapter]
\theoremstyle{remark}
\newtheorem*{remark}{Nota}
\newtheorem*{solution}{Solución}
\let\BeginKnitrBlock\begin \let\EndKnitrBlock\end
\begin{document}
\maketitle

%\cleardoublepage\newpage\thispagestyle{empty}\null
%\cleardoublepage\newpage\thispagestyle{empty}\null
%\cleardoublepage\newpage
\thispagestyle{empty}

\setlength{\abovedisplayskip}{-5pt}
\setlength{\abovedisplayshortskip}{-5pt}

{
\hypersetup{linkcolor=black}
\setcounter{tocdepth}{2}
\tableofcontents
}
\listoftables
\listoffigures
\chapter*{Prefacio}\label{prefacio}


\includegraphics{images/by-nc-sa.png}\\
La versión en línea de este libro se comparte bajo la licencia
\href{http://creativecommons.org/licenses/by-nc-sa/4.0/}{Creative
Commons Attribution-NonCommercial-ShareAlike 4.0 International License}.

\section*{¿Por qué leer este libro?}\label{por-que-leer-este-libro}


Este libro es el resultado de enfocarnos en proveer la mayor cantidad de
material sobre Probabilidad y Estadística Matemática con un desarrollo
teórico lo más explícito posible, con el valor agregado de incorporar
ejemplos de las finanzas y la programación en \texttt{R}. Finalmente
tenemos un libro interactivo que ofrece una experiencia de aprendizaje
distinta e innovadora.

Es mucha la literatura, pero son pocas las opciones donde se pueda
navegar el libro de forma amigable y además contar con ejemplos en
\texttt{R} y ejercicios interactivos, además del contenido multimedia.
Esperamos que ésta sea un contribución sobre nuevas prácticas para
publicar el contenido y darle vida, crear una experiencia distinta, una
experiencia interactiva y visual. El reto es darle vida al contenido
asistidos con las herramientas de Internet.

Finalmente este es un intento de ofrecer otra visión sobre la enseñanza
y la generación de material más accesible. Estamos en un mundo
multidisciplinado, es por ello que ahora hay que generar contenido que
conjugue en un mismo lugar las matemáticas, estadística, finanzas y la
computación.

Lo dejamos público ya que las herramientas que usamos para ensamblarlo
son abiertas y públicas.

\section*{Estructura del libro}\label{estructura-del-libro}


\section*{Información sobre los programas y
convenciones}\label{informacion-sobre-los-programas-y-convenciones}
\addcontentsline{toc}{section}{Información sobre los programas y
convenciones}

Este libro es posible gracias a una gran cantidad de desarrolladores que
contribuyen en la construcción de herramientas para generar documentos
enriquecidos e interactivos. En particular al autor de los paquetes
Yihui Xie xie2015.

\section*{Prácticas interactivas con
R}\label{practicas-interactivas-con-r}


Vamos a utilizar el paquete
\href{https://github.com/datacamp/tutorial}{Datacamp Tutorial} que
utiliza la librería en JavaScript
\href{https://github.com/datacamp/datacamp-light}{Datacamp Light} para
crear ejercicios y prácticas con \texttt{R}. De esta forma el libro es
completamente interactivo y con prácticas incluidas. De esta forma
estamos creando una experiencia única de aprendizaje en línea.

eyJsYW5ndWFnZSI6InIiLCJwcmVfZXhlcmNpc2VfY29kZSI6ImIgPC0gNSIsInNhbXBsZSI6IiMgQ3JlYSB1bmEgdmFyaWFibGUgYSwgaWd1YWwgYSA1XG5cblxuIyBNdWVzdHJhIGVsIHZhbG9yIGRlIGEiLCJzb2x1dGlvbiI6IiMgQ3JlYSB1bmEgdmFyaWFibGUgYSwgaWd1YWwgYSA1XG5hIDwtIDVcblxuIyBNdWVzdHJhIGVsIHZhbG9yIGRlIGFcbmEiLCJzY3QiOiJ0ZXN0X29iamVjdChcImFcIilcbnRlc3Rfb3V0cHV0X2NvbnRhaW5zKFwiYVwiLCBpbmNvcnJlY3RfbXNnID0gXCJBc2VnJnVhY3V0ZTtyYXRlIGRlIG1vc3RyYXIgZWwgdmFsb3IgZGUgYGFgLlwiKVxuc3VjY2Vzc19tc2coXCJFeGNlbGVudGUhXCIpIn0=

\section*{Agradecimientos}\label{agradecimientos}


A todo el equipo de Synergy Vision que no deja de soñar. Hay que hacer
lo que pocos hacen, insistir, insistir hasta alcanzar. Lo más importante
es concretar las ideas. La idea es sólo el inicio y solo vale cuando se
concreta.

\BeginKnitrBlock{flushright}
Synergy Vision, Caracas, Venezuela
\EndKnitrBlock{flushright}

\chapter*{Acerca del Autor}\label{acerca-del-autor}


Este material es un esfuerzo de equipo en Synergy Vision,
(\url{http://synergy.vision/nosotros/}).

El propósito de este material es ofrecer una experiencia de aprendizaje
distinta y enfocada en el estudiante. El propósito es que realmente
aprenda y practique con mucha intensidad. La idea es cambiar el modelo
de clases magistrales y ofrecer una experiencia más centrada en el
estudiante y menos centrado en el profesor. Para los temas más técnicos
y avanzados es necesario trabajar de la mano con el estudiante y
asistirlo en el proceso de aprendizaje con prácticas guiadas, material
en línea e interactivo, videos, evaluación contínua de brechas y
entendimiento, entre otros, para procurar el dominio de la materia.

Nuestro foco es la Ciencia de los Datos Financieros y para ello se
desarrollará material sobre: \textbf{Probabilidad y Estadística
Matemática en R}, \textbf{Programación Científica en R},
\textbf{Mercados}, \textbf{Inversiones y Trading}, \textbf{Datos y
Modelos Financieros en R}, \textbf{Renta Fija}, \textbf{Inmunización de
Carteras de Renta Fija}, \textbf{Teoría de Riesgo en R},
\textbf{Finanzas Cuantitativas}, \textbf{Ingeniería Financiera},
\textbf{Procesos Estocásticos en R}, \textbf{Series de Tiempo en R},
\textbf{Ciencia de los Datos}, \textbf{Ciencia de los Datos
Financieros}, \textbf{Simulación en R}, \textbf{Desarrollo de
Aplicaciones Interactivas en R}, \textbf{Minería de Datos},
\textbf{Aprendizaje Estadístico}, \textbf{Estadística Multivariante},
\textbf{Riesgo de Crédito}, \textbf{Riesgo de Liquidez}, \textbf{Riesgo
de Mercado}, \textbf{Riesgo Operacional}, \textbf{Riesgo de Cambio},
\textbf{Análisis Técnico}, \textbf{Inversión Visual}, \textbf{Finanzas},
\textbf{Finanzas Corporativas}, \textbf{Valoración}, \textbf{Teoría de
Portafolio}, entre otros.

Nuestra cuenta de Twitter es (\url{https://twitter.com/bysynergyvision})
y nuestros repositorios están en GitHub
(\url{https://github.com/synergyvision}).

\textbf{Somos Científicos de Datos Financieros}

\mainmatter

\chapter{Introducción}\label{introduccion}

\chapter{Estructuras algebraicas}\label{estructuras-algebraicas}

\section{Conjuntos}\label{conjuntos}

Abordaremos los temas relacionados con la teoría de conjuntos desde una
perspectiva intuitiva, más bien oparacional para abordar los conceptos
básicos necesarios para desarrollar el resto de los capítulos.

\subsection{Definiciones Iniciales}\label{definiciones-iniciales}

Entenderemos por \textbf{conjunto} a una colección de objetos
cualesquiera. Las palabras \emph{conjunto, colecci'on, familia} suelen
ser usadas para denotar este objeto matemático. Los objetos que
conforman un conjunto son llamados comunmente \textbf{elementos} (o
\textbf{miembros}) del conjunto. Los conjuntos suelen denotarse con
letras mayúsculas \(A, B, C, P,\cdots\), mientras que los miembros que
los conforman generalmente se denotan con letras minúsculas
\(a,b, x, y,\cdots\). Si \(C\) es un conjunto y \(x\) es un elemento de
\(C\), escribiremos \(x\in C\) (o equivalentemente \(C\ni x\)) lo que se
lee \emph{\(x\) pertenece al conjunto \(C\)}. Para denotar lo contrario
usaremos el símbolo \(\notin\), así \(x\notin C\) significa que
\emph{\(x\) no pertenece a \(C\) o \(x\) no es miembro de \(C\)}.

\smallskip

Puede ocurrir que elementos de un conjunto también pertenezcan a otro
conjunto. En el caso que todo elemento de un conjunto \(A\) es miembro
del conjunto \(C\) decimos que \textbf{\(A\) es subconjunto de \(C\)} (o
\(C\) contiene a \(A\)), lo que denotaremos \(A\subseteq C\). Es decir,
si \(x\in A\), entonces \(x\in C\) para todo \(x\), implica que
\(A\subseteq C\). Note que es posible que \(A\) y \(C\) tengan
exactamente los mismos elementos, en este caso diremos que los conjuntos
\(A\) y \(C\) son iguale y lo denotaremos por \(A=C\). Sin embargo
debemos comprobar que \(A\subseteq C\) y \(C\subseteq A\) para asegurar
que \(A=C\). En otro caso, cuando \(A\subseteq C\) pero \(A\) no es
igual al conjunto \(C\), diremos que \textbf{\(A\) es un subconjunto
propio de \(C\)} (o \textbf{\(A\) está propiamente contenido en \(C\)}).

\BeginKnitrBlock{example}
\protect\hypertarget{exm:unnamed-chunk-8}{}{\label{exm:unnamed-chunk-8} }El
conjunto formado por 1, 2, 3, 4, etc. es el llamado conjunto de los
\textbf{números naturales} y se denota por \(\mathbb{N}\).

Se debe saber que podemos definir un conjunto describiendo uno a uno sus
miembros. Esto se hace encerrándolos entre llaves. Así, el conjunto de
los números naturales es \(\mathbb{N}=\{1,2,3,4,\cdots\}\).
\EndKnitrBlock{example}

\BeginKnitrBlock{example}
\protect\hypertarget{exm:unnamed-chunk-9}{}{\label{exm:unnamed-chunk-9}
}Dado el conjunto de los números enteros
\(\mathbb{Z}=\{\cdots,-3,-2,-1,0,1,2,3,\cdots \}\), el conjunto de los
números pares (enteros pares) es el conjunto de los números de la forma
\(2k\) donde \(k\) es un entero.

También se puede describir el conjunto anterior así:

\[\{p\in\mathbb{Z}| p=2k \mbox{ para algún } k\in\mathbb{Z} \}\]

lo cual se lee: \emph{el conjunto formado por todos los números enteros}
\(p\) \emph{tales que} \(p=2k\) \emph{para algún número entero} \(k\).
\EndKnitrBlock{example}

\BeginKnitrBlock{example}
\protect\hypertarget{exm:unnamed-chunk-10}{}{\label{exm:unnamed-chunk-10}
}Denotaremos el conjunto de los números reales po \(\mathbb{R}\). El
conjunto de las soluciones de la ecuación \(7x^{2}+4x-32=0\) es
\(C=\{x\in\mathbb{R}| x \mbox{ es solución de } 7x^{2}+4x-32=0 \}\)
\EndKnitrBlock{example}

\subsection{Operaciones entre
conjuntos}\label{operaciones-entre-conjuntos}

Dados dos conjuntos \(A\) y \(B\) podemos definir nuevos conjuntos a
partir de estos, con las operaciones entre conjuntos que definiremos a
continuación.

\BeginKnitrBlock{definition}
\protect\hypertarget{def:uniondeconjuntos}{}{\label{def:uniondeconjuntos}
}Dados dos conjuntos \(A\) y \(B\), el conjunto \textbf{unión} de \(A\)
y \(B\) es el conjunto \(A\cup B = \{ x| x\in A \mbox{ o } x\in B \}\).
\EndKnitrBlock{definition}

Es decir, la unión de los conjuntos \(A\) y \(B\) es el conjunto
formados por aquellos elementos que pertenezcan a al menos uno de los
dos conjuntos, así un elemento que pertenezca tanto a \(A\) como a
\(B\), es miembro de \(A\cup B\).

\BeginKnitrBlock{definition}
\protect\hypertarget{def:intersecciondeconjuntos}{}{\label{def:intersecciondeconjuntos}
}Dados dos conjuntos \(A\) y \(B\), el conjunto \textbf{intersección} de
\(A\) y \(B\) es el conjunto
\(A\cap B = \{ x| x\in A \mbox{ y } x\in B \}\).
\EndKnitrBlock{definition}

En otras palabras, la intersección de \(A\) y \(B\) es el conjunto
formado por aquellos elementos que pertenecen a ambos conjuntos
simultaneamente.

\BeginKnitrBlock{example}
\protect\hypertarget{exm:unnamed-chunk-11}{}{\label{exm:unnamed-chunk-11}
}Dados los conjuntos \(A\) y \(B\), tales que \(B\subseteq A\) (\(B\) es
subconjunto de \(A\)). Se tiene que \(A\cup A=A\), más aún
\(A\cup B=A\). Además \(A\cap A=A\) y \(A\cap B=B\).
\EndKnitrBlock{example}

\BeginKnitrBlock{example}
\protect\hypertarget{exm:unnamed-chunk-12}{}{\label{exm:unnamed-chunk-12}
}Sean \(A=\{a,b,c\}\) y \(B=\{c,d,e\}\). \(A\cup B=\{a,b,c,d,e\}\) y
\(A\cap B=\{c\}\).
\EndKnitrBlock{example}

\BeginKnitrBlock{example}
\protect\hypertarget{exm:unnamed-chunk-13}{}{\label{exm:unnamed-chunk-13}
}Dado el conjunto de los números enteros, \(\mathbb{Z}\). Los
subconjuntos
\(\mathbb{Z}^{+}=\{p\in\mathbb{Z}| p \mbox{ es un entero positivo} \}\)
y
\(\mathbb{Z}^{-}=\{p\in\mathbb{Z}| p \mbox{ es un entero negativo} \}\).
Se tiene que \(\mathbb{Z}\cap \mathbb{Z}^{+}=\mathbb{Z}^{+}\) y
\(\mathbb{Z}\cup \mathbb{Z}^{-}=\mathbb{Z}\).

Ahora bien, pensemos en el conjunto
\(\mathbb{Z}^{+}\cap\mathbb{Z}^{-}\). Note que no existe nímero entero
que pertenezca a \(\mathbb{Z}^{+}\) y \(\mathbb{Z}^{-}\)
simultaneamente. Para que la intersección esté bien definida, el
resultado debería ser un conjunto. Con ese fin daremos la siguiente
definición.
\EndKnitrBlock{example}

\BeginKnitrBlock{definition}
\protect\hypertarget{def:conjuntovacio}{}{\label{def:conjuntovacio} }Diremos
que un conjunto es \textbf{vacío} si no posee elementos y lo denotaremos
por \(\emptyset\).
\EndKnitrBlock{definition}

\BeginKnitrBlock{remark}
\iffalse{} {Nota. } \fi{}El conjunto vacío es único. Basta con demostrar
que dados dos conjuntos \(\emptyset\) y \(\emptyset_{1}\), se cumple que
\(\emptyset\subseteq \emptyset_{1}\) y
\(\emptyset_{1}\subseteq\emptyset\).
\EndKnitrBlock{remark}

\BeginKnitrBlock{definition}
\protect\hypertarget{def:conjuntodiferencia}{}{\label{def:conjuntodiferencia}
}Dados dos conjuntos \(A\) y \(B\), el \textbf{conjunto diferencia}
\(A-B\) es el conjunto \(\{x\in A | x\notin B \}\).
\EndKnitrBlock{definition}

\BeginKnitrBlock{example}
\protect\hypertarget{exm:unnamed-chunk-15}{}{\label{exm:unnamed-chunk-15}
}Dados los conjuntos de los números enteros, \(\mathbb{Z}\) y el
conjunto de los números naturales \(\mathbb{N}=\{0,1,2,3,4,\cdots \}\),
el conjunto diferencia
\(\mathbb{Z}-\mathbb{N}=\{p\in\mathbb{Z}| p\notin \mathbb{N} \}\) es
decir, el conjunto de los números enteros que no son números naturales,
que no es más que \(\mathbb{Z}^{-}\).
\EndKnitrBlock{example}

\medskip

\emph{Generalización de unión e intersección}

\smallskip

Las operaciones entre conjuntos definidas antes consideran solo dos
conjuntos, sin embargo podemos extender las nociones de unión e
intesección de conjuntos a una cantidad cualquiera de conjuntos, finita
o no.

Dado \(n\in\mathbb{N}\). La unión de \(n\) conjuntos
\(A_{1}, A_{2},\cdots,A_{n}\), puede definirse claramente a partir de la
unión de dos conjuntos así:

\begin{enumerate}
\def\labelenumi{\roman{enumi})}
\item
  Hallamos el conjunto unión de los primeros dos conjuntos
  \(A_{1}\cup A_{2}\) (definición \ref{def:uniondeconjuntos}).
\item
  Ahora unimos el conjunto obtenido en el paso i. con el siguiente
  conjunto \(A_{3}\), esto es \((A_{1}\cup A_{2})\cup A_{3}\).
\item
  Repetimos el paso anterior hasta unir todos los conjuntos.
\end{enumerate}

El algoritmo anterior nos muestra que es posible unir una cantidad
finita cualquiera de conjuntos. Analogamente se pueden intersecctar
\(n\) conjuntos, siendo \(n\) un número natural cualquiera.

Ahora bien, la unión e intersección de una cantidad arbitraria (no
necesariamente finita) también se puede definir.

Dado \(I\) un conjunto de índices. Una familia indexada por \(I\) es una
colección \(\mathcal{F}=\{A_{\alpha} | \, \alpha\in I \}\). Note que
esta definición permite que el conjunto \(I\) sea cualquier conjunto,
finito o infinito. Se suele usar el conjunto de los números enteros
positivos como conjunto de índices (para numerar) pero se puede usar
cualquier otro conjunto.

La \textbf{unión} de los conjuntos \(A_{\alpha}\), con \(\alpha\in I\),
el el conjunto
\(\bigcup\mathcal{F}=\bigcup_{\alpha\in I} A_{\alpha}=\{x|\, x\in A_{\alpha} \mbox{ para algún } \alpha\in I \}\).
De forma análoga, la \textbf{intersección} de los conjuntos
\(A_{\alpha}\), con \(\alpha\in I\), es el conjunto
\(\bigcap A_{\alpha}=\{x|\, x\in A_{\alpha} \mbox{ para todo } \alpha\in I \}\).
Vale resaltar que cuando \(I\) es un conjunto finito, digamos
\(I=\{1,2,\cdots, n\}\), se tiene que
\(\bigcup A_{\alpha}=\{x|\, x\in A_{\alpha} \mbox{ para algún } \alpha\in I \}=\{x|\, x\in A_{1} \mbox{ o } x\in A_{2}\mbox{ o }\cdots \mbox{ o } x\in A_{n} \}=A_{1}\cup A_{2},\cdots , A_{n}\).

\subsection{Producto cartesiano}\label{producto-cartesiano}

\BeginKnitrBlock{definition}
\protect\hypertarget{def:unnamed-chunk-16}{}{\label{def:unnamed-chunk-16}
}Dados \(a\) y \(b\) miembros de los conjuntos \(A\) y \(B\)
respectivamente, llamaremos *par ordenado al arreglo \((a,b)\). Diremos
que dos pares ordenados \((a,b)\) y \((c,d)\) son iguales (es decir
\((a,b)=(c,d)\) si y solo si \(a=c\) y \(b=d\).
\EndKnitrBlock{definition}

\BeginKnitrBlock{definition}
\protect\hypertarget{def:unnamed-chunk-17}{}{\label{def:unnamed-chunk-17}
}Sean \(A\) y \(B\) dos conjuntos. El \emph{producto cartesiano de \(A\)
y \(B\)} es el conjunto \(A\times B\) formado por todos los pares
ordenados \((a,b)\), donde \(a\in A\) y \(b\in B\). Es decir
\(A\times B=\{(a,b)| \, a\in A \mbox{ y } b\in B \}\).
\EndKnitrBlock{definition}

\BeginKnitrBlock{example}
\protect\hypertarget{exm:unnamed-chunk-18}{}{\label{exm:unnamed-chunk-18}
}Si \(A=\{1,2,3\}\) y \(B=\{3,4,5\}\), entonces el producto cartesiano
de \(A\) y \(B\) es el conjunto

\(A\times B=\{ (1,3), (1,4), (1,5), (2,3), (2,4), (2,5), (3,3), (3,4), (3,5)\}\)
\EndKnitrBlock{example}

\BeginKnitrBlock{remark}
\iffalse{} {Nota. } \fi{}Dados los conjuntos \(A\) y \(B\), podemos
considerar los conjuntos \(A\times B\) y \(B\times A\). Como conjuntos
son distintos, aunque existe una relación entre ellos (se discutirá mas
adelante cuando se definan correspondencias biyectivas ). De igual forma
se puede definir el producto cartesiano de una cantidad finita de
conjuntos (de forma análoga a la unión de conjuntos). Si \(A\), \(B\) y
\(C\) son conjuntos, podemos definir el conjunto \(A\times B\times C\)
formado por los arreglos (terna ordenada) del tipo \((a,b,c)\), donde
\(a\in A\), \(b\in B\) y \(c\in C\). Más aun, se puede definir el
producto cartesiano para una cantidad arbitraria de conjuntos. Dado un
conjunto de índices \(I\), el producto cartesiano de una familia
indexada por \(I\), \(\mathcal{F}=\{A_{\alpha}|\, \alpha\in I \}\), es
el conjunto de aplicaciones
\(f:I\longrightarrow \bigcup\mathcal{F} \mbox{ tales que } f(\alpha)\in A_{\alpha}\),
es decir
\(\prod_{\alpha \in I} A_{\alpha}=\{f:I\longrightarrow\bigcup \mathcal{F} |\, f(\alpha)\in A_{\alpha} \}\).
Por último, queremos recalcar que es posible hacer el producto
cartesiano de un conjunto consigo mismo, esto es \(A\times A\) formado,
como es natural por los pares \((a,b)\), con \(a\) y \(b\) elementos de
\(A\). En este caso se puede denotar \(A^{2}\) al producto cartesiano de
\(A\) consigo mismo (y se denota \(A^{n}\) al producto cartesiano de
\(A\) consigo mismo \(n\) veces). Para \(A^{2}\) llamamos
\textbf{diagonal} al conjunto formado por los pares de la forma
\((a,a)\) y se denota por \(\Delta\), es decir,
\(\Delta=\{(a,a)|\, a\in A \}\)
\EndKnitrBlock{remark}

\section{Relaciones de Equivalencia}\label{relaciones-de-equivalencia}

\smallskip

\BeginKnitrBlock{definition}
\protect\hypertarget{def:unnamed-chunk-20}{}{\label{def:unnamed-chunk-20}
}Dados dos conjuntos \(A\) y \(B\), una \textbf{relación de \(A\) sobre
\(B\)}, es un subconjunto \(R\) del producto cartesiano \(A\times B\).
En el caso que \(R\subseteq A\times A\), se dice que \(R\) es una
relación sobre \(A\).
\EndKnitrBlock{definition}

\BeginKnitrBlock{definition}
\protect\hypertarget{def:unnamed-chunk-21}{}{\label{def:unnamed-chunk-21}
}Dada un relación \(R\) sobre un conjunto no vacío \(A\). Decimos que:

\begin{enumerate}
\def\labelenumi{\roman{enumi})}
\item
  \(R\) es \textbf{reflexiva} si para todo \(a\in A\), se tiene que
  \((a,a)\in R\) (es decir, la diagonal \(\Delta\) es subconjunto de
  \(R\)).
\item
  \(R\) es \textbf{simétrica} si para todo \(a,b \in A\), se cumple que:
  \((a,b)\in R \Leftrightarrow (b,a)\in R\).
\item
  \(R\) es \textbf{transitiva} si para todo \(a,b, c \in A\), se cumple
  que: \((a,b)\in R\) y \((b,c)\in R\) \(\Rightarrow (a,c)\in R\).

  \EndKnitrBlock{definition}
\end{enumerate}

\BeginKnitrBlock{definition}
\protect\hypertarget{def:unnamed-chunk-22}{}{\label{def:unnamed-chunk-22}
}Una relación \(R\) sobre un conjunto no vacío \(A\), que es reflexiva,
simétrica y transitiva, se dice \textbf{de equivalencia}.
\EndKnitrBlock{definition}

\BeginKnitrBlock{remark}
\iffalse{} {Nota. } \fi{}Si \(R\) es una relación sobre \(A\), el hecho
de que \((a,b)\in R\) se puede denotar como \(aRb\) o como \(a\simeq b\)
cuando el contexto lo permita y quede claro cual es la relación.

De este modo, \(R\) es una relación de equivalencia si para todo
\(a,b,c\in A\) se tiene que:

\begin{enumerate}
\def\labelenumi{\roman{enumi})}
\item
  \(aRa\),
\item
  \(aRb\Leftrightarrow bRa\) y
\item
  \(aRb\) y \(bRc\Rightarrow aRc\).

  \EndKnitrBlock{remark}
\end{enumerate}

\BeginKnitrBlock{example}
\protect\hypertarget{exm:unnamed-chunk-24}{}{\label{exm:unnamed-chunk-24}
}\textbf{La igualdad} es una relación de equivalencia.

Dado el conjunto de los números reales \(\mathbb{R}\), la igualdad es la
relación \(aRb\) siempre que \(a\) y \(b\) sean el mismo número real y
se denota por \(a=b\). Es claramente reflexiva ya que \(a=a\); simétrica
pues \(a=b\) implica que \(a\) y \(b\) son el mismo número real y por lo
tanto \(b=a\) y es transitiva ya que si \(a=b\) y \(b=a\), entonces
\(a,b\) y \(c\) son el mismo número real, lo que implica que \(a=c\).

Note que el conjunto donde definimos la igualdad (los reales
\(\mathbb{R}\)) es irrelevante. Esta relación puede (y de hecho está)
definida sobre cualquien conjunto.
\EndKnitrBlock{example}

\BeginKnitrBlock{example}
\protect\hypertarget{exm:unnamed-chunk-25}{}{\label{exm:unnamed-chunk-25}
}Dado el conjunto de los números enteros \(\mathbb{Z}\), definamos la
siguiente relación \(R\): \((a,b)\in R\) si y solo si \(b-a\) es un
número par (o lo que es igual, es un múltiplo de \(2\)).

\(R\) es una relación reflexiva, ya que \(a-a=0\) y cero es múltiplo de
\(2\). Es simétrica ya que si \(b-a\) es un número par, entonces \(a-b\)
también lo es. Ahora mostremos que es transitiva, sean \(a, b\) y \(c\)
números enteros tales que \((a,b)\in R\) y \((b,c)\in R\), entonces
\(b-a\) y \(c-b\) son números pares como también su suma
\(c-a=(c-b)+(b-a)\), por lo tanto \((a,c)\in R\). En consecuencia, \(R\)
es una relación de equivalencia.
\EndKnitrBlock{example}

\BeginKnitrBlock{example}
\protect\hypertarget{exm:unnamed-chunk-26}{}{\label{exm:unnamed-chunk-26}
}\textbf{La congruencia modular}. Definiremos la siguiente relación
sobre el conjunto de los números enteros, \(\mathbb{Z}\). Dado un número
natural \(n\), para cuales quiera enteros \(a\) y \(b\), decimos que
\(aRb\) si y solo si \(b-a\) es divisible por \(n\) (o equivalentemente,
\(b-a\) es múltiplo de \(n\)); lo denotamos por
\(a\cong b\mbox{ mod } n\) y se lee \emph{\(a\) congruente con \(b\)
módulo \(n\)}.

Es fácil demostrar que la relación es reflexiva.

Supongamos que \(a\cong b \mbox{ mod } n\), entonces \(b-a=kn\) para
algún \(k\in\mathbb{Z}\). Luego, \(a-b=-kn\) por lo tanto
\(b\cong a \mbox{ mod } n\). Entonces la relación es simétrica.

Ahora supongamos que \(a\cong b \mbox{ mod } n\) y
\(b\cong c \mbox{ mod } n\). Se tiene así que existen
\(k,q\in\mathbb{Z}\) tales que \(b-a=kn\) y \(c-b=qn\). De este modo
\(c-a=(c-b)+(b-a)=(q-k)n\), de donde se sigue que
\(a\cong c \mbox{ mod } n\).

De lo anterior se sigue que la relación de congruencia módulo \(n\) es
una relación de equivalencia.
\EndKnitrBlock{example}

\BeginKnitrBlock{definition}
\protect\hypertarget{def:unnamed-chunk-27}{}{\label{def:unnamed-chunk-27}
}Dados una relación de equivalencia \(R\) sobre un conjunto \(A\) y
\(a\in A\). Definimos \textbf{la clase de equivalencia de \(a\)} como el
conjunto \([a]=\{b\in A | aRb\}\). También se denota por \(cl(a)\) o
\(\tilde{a}\).
\EndKnitrBlock{definition}

Pensemos en las clases de equivalencias de los ejemplos anteriores. La
clase de equivalencia de \(a\) para la relación \emph{igualdad} es el
conjunto cuyo único elemento es \(a\). Mientras que en el segundo
ejemplo, la relación solo define dos clases de equivalencia, el conjunto
de los números enteros pares y el conjunto de los números enteros
impares. En el caso de la \emph{congruencia módulo \(n\)}, la clase de
equivalencia de un entero \(a\) es el conjunto
\(\{b\in\mathbb{Z}| a\cong b\mbox{ mod } n \}=\{b\in\mathbb{Z}| b-a=kn\mbox{ para algún } k\in\mathbb{Z} \}=\{b\in\mathbb{Z}| b=kn+a\mbox{ para algún } k\in\mathbb{Z} \}\)
es decir, todos los enteros \(b\) que tienen por resto \(a\) al ser
divididos por \(n\).

\smallskip

\BeginKnitrBlock{definition}
\protect\hypertarget{def:unnamed-chunk-28}{}{\label{def:unnamed-chunk-28}
}Dado un conjunto \(A\), \textbf{una partición de \(A\)} es una
colección de subconjuntos no vacíos de \(A\), disjuntos dos a dos, tales
que la unión de ellos es todo \(A\). Es decir,
\(\{ B_{i}\subseteq A |\,, i\in I\}\), donde \(I\) es un conjunto de
índices y se tiene que:

\begin{enumerate}
\def\labelenumi{\roman{enumi})}
\item
  \(B_{i}\neq\emptyset\) para todo \(i\in I\).
\item
  \(B_{i}\neq B_{j}\), para \(i,j\in I\) y \(i\neq j\).
\item
  \(\bigcup_{i\in I} B_{i}=A\).
\end{enumerate}

Cada subconjunto \(B_{i}\) es una parte de \(A\).
\EndKnitrBlock{definition}

\smallskip

\BeginKnitrBlock{theorem}
\protect\hypertarget{thm:unnamed-chunk-29}{}{\label{thm:unnamed-chunk-29}
}Las clases de equivalencia definidas por una relación de equivalencia
sobre un conjunto \(A\) definen una partición de \(A\). Recíprocamente,
una partición de un conjunto \(A\), induce una relación de equivalencia
sobre \(A\) de forma que las clases de equivalencia corresponden a las
partes de la partición.
\EndKnitrBlock{theorem}

\smallskip

\BeginKnitrBlock{proof}
\iffalse{} {Demostración. } \fi{}Sea \(R\) una clase de equivalencia
sobre \(A\). Por la reflexividad de \(R\) se tiene que \(a\in[a]\), por
lo tanto \(\bigcup_{a\in A} [a]= A\) y \([a]\) es no vacío. Supongamos
que \([a]\cap [b]\neq\emptyset\), entonces existe \(c\in A\) tal que
\(c\in [a]\cap [b]\), por transitividad, \(aRb\), por lo tanto
\([a]=[b]\), es decir, si dos clases no son disjuntas, son iguales.

Recíprocamente, sea \(\{ B_{i}\subseteq A |\,, i\in I\}\) una partición.
Definimos la relación \(R\), sobre \(A\), así: para \(a,b\in A\),
\(aRb\) si y solo si existe \(i\in I\) tal que \(a,b\in B_{i}\), esto
es, \(a\) y \(b\) pertenecen a la misma parte \(B_{i}\). Es muy fácil
ver que esta relación es de equivalencia.
\EndKnitrBlock{proof}

\section{Funciones}\label{funciones}

\smallskip

Veamos ahora la definición de función (o aplicación), concepto
importantísimo en toda la matemática y bastante conocido y usado en la
educación matemáticas desde los niveles más básicos. Digamos que una
\emph{funci'on} es una regla de asignación entre conjuntos, por ejemplo
la función que asigna a cada número real \(r\) su parte entera
\(\lVert n \rVert\) (el mayor entero menor o igual que \(r\)), es una
función del conjunto de los números reales \(\mathbb{R}\) al conjunto de
los números enteros y su regla de asignación es la antes descrita. La
relación \(y=x^{2}\), es una función de \(\mathbb{R}\) en si mismo que a
cada número real \(x\) le relaciona su cuadrado \(x^{2}\), desde otro
punto de vista, los pares \((x,y)\) pertenecen a la función si
\(y=x^{2}\), dicho de otro modo, los pares \((x,x^{2})\) forman parte de
la función.

\BeginKnitrBlock{definition}
\protect\hypertarget{def:unnamed-chunk-31}{}{\label{def:unnamed-chunk-31}
}Dados dos conjuntos no vacíos \(A\) y \(B\), una \textbf{función de
\(A\) en \(B\)} es un subconjunto \(G\) de \(A\times B\) tal que para
cada \(a\in A\), existe un único \(b\in B\), tal que \((a,b)\in G\). Lo
denotamos por \(f: A\longrightarrow B\), con \(f(a)=b\). Llamaremos
\textbf{dominio de \(f\)} al conjunto \(A\) (y se denota \(dom(f\)) y
\textbf{codominio} al conjunto \(B\). También se suelen llamar conjunto
de partida y conjunto de llegada respectivamente.
\EndKnitrBlock{definition}

\BeginKnitrBlock{example}
\protect\hypertarget{exm:identidad}{}{\label{exm:identidad} }Dado un
conjunto \(A\) no vacío, la \textbf{función identidad} es aquella que a
cada \(a\), le asigna el mismo elemento \(a\). Esto es,
\(i: A\longrightarrow A\) definida por \(i(a)=a\) para todo \(a\in A\),
que no es más que la diagonal de \(A\times A\).
\EndKnitrBlock{example}

\BeginKnitrBlock{example}
\protect\hypertarget{exm:ejm1-12}{}{\label{exm:ejm1-12} }Sea
\(f: \mathbb{R}\longrightarrow\mathbb{R}\) definida por \(f(x)=x^{2}\).
Los pares de la forma \((x,x^{2})\) forman parte de la función. El
conjunto \(dom(f)=\mathbb{R}\) y su codominio es \(\mathbb{R}\).
\EndKnitrBlock{example}

\BeginKnitrBlock{example}
\protect\hypertarget{exm:ejm1-13}{}{\label{exm:ejm1-13} }Dados
\(\mathbb{Z}\) el conjunto de los números enteros, y \(\mathbb{Q}\) el
conjunto de los números racionales, definimos
\(f:\mathbb{Z}\times\mathbb{Z}\longrightarrow\mathbb{Q}\) de la
siguiente forma \(f((m,n))=\frac{m}{n}\).
\EndKnitrBlock{example}

\BeginKnitrBlock{example}
\protect\hypertarget{exm:ejm1-14}{}{\label{exm:ejm1-14} }Dados
\(\mathbb{Z}\) el conjunto de los números enteros, y \(\mathbb{Q}\) el
conjunto de los números racionales, definimos
\(f:\mathbb{Q}\longrightarrow \mathbb{Z}\times\mathbb{Z}\) de la
siguiente forma. Dado un número racional \(q\in\mathbb{Q}\), existen
enteros sin factores comunes \(m\) y \(n\) tales que \(q=\frac{m}{n}\).
Así, \(f(q)=(m,n)\).
\EndKnitrBlock{example}

\BeginKnitrBlock{example}
\protect\hypertarget{exm:ejm1-15}{}{\label{exm:ejm1-15} }Sean \(A\) y \(B\)
conjuntos no vacíos tales que \(A\subseteq B\),
\(\imath: A \longrightarrow B\) dada por \(\imath(a)=a\) es la función
\textit{inclusión de $A$ en $B$}.
\EndKnitrBlock{example}

\BeginKnitrBlock{example}
\protect\hypertarget{exm:ejm1-16}{}{\label{exm:ejm1-16} }Sea \(C\) el
conjunto \(\{a,b,c\}\). Podemos definir la siguiente función
\(f:C\longrightarrow C\), donde \(f(a)=b\), \(f(b)=c\) y \(f(c)=a\).
\EndKnitrBlock{example}

\BeginKnitrBlock{example}
\protect\hypertarget{exm:ejm1-17}{}{\label{exm:ejm1-17} }Sean \(A\) y \(B\)
conjuntos no vacíos. \(\pi: A\times B\longrightarrow A\) dada por
\(\pi((a,b))=a\) es la función
\textit{proyección de $A\times B$ sobre $A$}. Se puede definir de forma
análoga la proyección sobre \(B\).
\EndKnitrBlock{example}

\BeginKnitrBlock{example}
\protect\hypertarget{exm:ejm1-18}{}{\label{exm:ejm1-18} }Dado un conjunto
\(C\) sobre el cual está definida una relación de equivalencia \(R\).
Llamaremos \textbf{conjunto cociente} al conjunto de las clases de
equivalencia definidas por \(R\), esto es \(C/R=\{[a] | a\in C \}\).
Definimos \(f:C\longrightarrow C/R\), como \(f(a)=[a]\).
\EndKnitrBlock{example}

\BeginKnitrBlock{definition}
\protect\hypertarget{def:unnamed-chunk-32}{}{\label{def:unnamed-chunk-32}
}Una función \(f:A\longrightarrow B\) se dice \textbf{sobreyectiva} si
para todo \(b\in B\), existe \(a\in A\) tal que \(f(a)=b\).
\EndKnitrBlock{definition}

Otra forma de entender la sobreyectividad es pensar que la función
``cubre'' todo el conjunto de llegada (el codominio).\textbackslash{}

También podemos entenderla en términos de la imagen de la función, que
definiremos a continuación: dada \(f:A\longrightarrow B\), si
\(C\subseteq A\), definimos la \textbf{imagen de un conjunto \(C\)} como
el conjunto \(\{f(a) | a\in C \}\), al cual denotamos por \(f[C]\) o
\(f''C\). El conjunto imagen del dominio se llamará \textbf{Imagen de
\(f\)} a secas, este es \(f[A]\) y también se denota \(Img(f)\).
Entonces la sobreyectividad es equivalente a que la imagen de \(f\) sea
igual al codominio, es decir \(f[A]=B\).

\BeginKnitrBlock{definition}
\protect\hypertarget{def:unnamed-chunk-33}{}{\label{def:unnamed-chunk-33}
}Una función \(f:A\longrightarrow B\) se llamará \textbf{inyectiva} si
para todo \(a\mbox{ y }b\in A\), si \(a\neq b\), entonces
\(f(a)\neq f(b)\).
\EndKnitrBlock{definition}

Un función inyectiva es pues una función que a cada elemento del dominio
le asocia elementos distintos del codominio. Para entender mejor esta
definición, definiremos imagen inversa: dada \(f:A\longrightarrow B\),
si \(C\subseteq B\), definimos la \textbf{imagen inversa de un conjunto
\(C\)} como el conjunto \(\{a | f(a)\in C \}\) y se denota por
\(f^{-1}[C]\). De este modo, una función es inyectiva si la imagen
inversa de los subconjuntos unitarios del codominio tienen a lo sumo un
elemento, es decir, \(f^{-1}[\{b\}]\) tiene un elemento o es vacío, para
todo \(b\in B\).

En los ejemplos que antes vimos, la función identidad es sobreyectiva e
inyectiva, es decir \textbf{biyectiva} (cuando una función es inyectiva
y sobreyectiva se le llama biyectiva). La función del ejemplo
\ref{exm:ejm1-12} no es inyectiva, basta ver que \(f(1)=f(-1)\). Tampoco
es sobreyectiva, no existe número real que tenga un cuadrado negativo.
Los ejemplos \ref{exm:ejm1-15} y \ref{exm:ejm1-16} muestran funciones
biyectivas. Pero los ejemplos \ref{exm:ejm1-13} y \ref{exm:ejm1-18} son
funciones sobreyectivas que no son inyectivas, mientras que el ejemplo
\ref{exm:ejm1-15} muestra una función inyectiva que no es sobreyectiva.

A continuación definirimos cuando dos funciones son iguales.
Intiutivamente, dos funciones serán iguales cuando expresen la misma
regla de asignación sobre los mismos objetos. En seguida la definición
formal:

\BeginKnitrBlock{definition}
\protect\hypertarget{def:unnamed-chunk-34}{}{\label{def:unnamed-chunk-34}
}Dos funciones \(f\) y \(g\) de \(A\) en \(B\), se dicen que son
\textbf{iguales} si \(f(a)=g(a)\) para todo \(a\in A\).
\EndKnitrBlock{definition}

Podemos también plantear la situación en la que se relacionen los
elementos de dos conjuntos pasando por un tercer conjunto haciendo uso
de dos funciones. Es decir, una regla de asignación entre los elementos
de un conjunto \(A\), en otro conjunto \(B\), y otra regla que relacione
a los elementos de \(B\) con un conjunto \(C\) se pueden componer para
obtener una regla (una función) de \(A\) a \(C\).

\BeginKnitrBlock{definition}
\protect\hypertarget{def:unnamed-chunk-35}{}{\label{def:unnamed-chunk-35}
}Sean \(f:A\longrightarrow B\) y \(g:B\longrightarrow C\) dos funciones.
La composición de \(f\) y \(g\) es una función de \(A\) en \(B\) que
asigna a cada \(a\in A\) el elemento \(g(f(a))\in C\). Se denota por
\(g\circ f\). Entonces \(g\circ f: A\longrightarrow B\), definido por
\((g\circ f)(a)=g(f(a))\).
\EndKnitrBlock{definition}

\BeginKnitrBlock{remark}
\iffalse{} {Nota. } \fi{}Note que el dominio de la función \(g\) (la
segunda en ser aplicada) debe ser igual al codominio de la función \(f\)
(pudiése ser un subconjunto del codominio).

Es importante el orden de las funciones, en el contexto general descrito
en la definición, no tiene sentido pensar en la composición
\(f\circ g\), ya que \(g(b)\) es un elemento del conjunto \(C\) que no
es el dominio de \(f\), por lo tanto la expresión \(f(g(b))\) carece de
sentido, salvo que \(B\) sea subconjunto de \(A\).
\EndKnitrBlock{remark}

\BeginKnitrBlock{example}
\protect\hypertarget{exm:unnamed-chunk-37}{}{\label{exm:unnamed-chunk-37}
}Sea \(A=\{a,b,c\}\). Sean \(f:A\longrightarrow A\) y
\(g:A\longrightarrow A\) funciones definida por \(f(a)=b\), \(f(b)=c\) y
\(f(c)=a\) y \(g(a)=a\), \(g(b)=c\) y \(g(c)=b\). Entonces
\((g\circ f)(a)=g(f(a))=g(b)=c\), \((g\circ f)(b)=g(f(b))=g(c)=b\) y
\((g\circ f)(c)=g(f(c))=g(a)=a\). Análogamente, \((f\circ g)(a)=b\),
\((f\circ g)(b)=a\) y \((f\circ g)(c)=c\). Aunque ambas funciones
compuestas, \(g\circ f\) y \(f\circ g\) son funciones de \(A\) en si
mismo, no son iguales ya que \((g\circ f)(a)=c\) y \((f\circ g)(a)=b\),
es decir \((g\circ f)(a)\neq (f\circ g)(a)\)
\EndKnitrBlock{example}

\BeginKnitrBlock{example}
\protect\hypertarget{exm:unnamed-chunk-38}{}{\label{exm:unnamed-chunk-38}
}Sea \(A=\{a,b,c\}\). Sean \(f:A\longrightarrow A\) y
\(g:A\longrightarrow A\) funciones definida por \(f(a)=b\), \(f(b)=c\) y
\(f(c)=a\) y \(g(b)=a\), \(g(c)=b\) y \(g(a)=c\). Entonces \(g\circ f\)
(y \(f\circ g\)) es la función identidad de \(A\).
\EndKnitrBlock{example}

\BeginKnitrBlock{example}
\protect\hypertarget{exm:unnamed-chunk-39}{}{\label{exm:unnamed-chunk-39}
}Dadas las siguientes funciones
\(f:\mathbb{R}\longrightarrow \mathbb{Z}:x\longmapsto \lVert x\rVert\),
la notación que sigue de los dos puntos, \(x\longmapsto \lVert x\rVert\)
nos indica la regla de asignación, esto es \(f(x)=\lVert x\rVert\),
donde \(\lVert x\rVert\) denota la \textit{parte entera} del número real
\(x\), a saber: \textit{el mayor entero menor o igual a $x$}. Y la
función \(g:\mathbb{Z}\longrightarrow \{0,e\}\), definida por

\begin{equation}
    g(p) = \left\{
    \begin{array}{ll}
    0      & \mbox{ si } p \mbox{ es un número par } \\
    e      & \mbox{ si } p \mbox{ es un número impar }
 \end{array}
    \right.
\end{equation}

Entonces, la función \(g\circ f\) aplica números reales en \(\{0,e\}\).
Sin embargo \(f\circ g\) no puede definirse, ya que el codominio de
\(g\) no es un subconjunto del dominio de \(f\).

Calculemos \(g\circ f\) para algunos números:
\((g\circ f)(\frac{1}{2})=g(\lVert \frac{1}{2} \rVert)=g(0)=0\),
\((g\circ f)(\frac{-3}{2})=g(\lVert \frac{-3}{2} \rVert)=g(-2)=0\) y
\((g\circ f)(\pi)=g(\lVert \pi \rVert)=g(3)=e\).
\EndKnitrBlock{example}

Al igual que se pueden componer dos funciones, \(f\) y \(g\), también se
puede hacer con una cantidad cualquiera (finita) de funciones. Dadas las
funciones \(f:A\longrightarrow B\), \(g:B\longrightarrow C\) y
\(h:\longrightarrow D\), podemos componer \(f\) y \(g\) y obtener una
función de \(A\) en \(C\). Y a su vez, componer esta función (la
compuesta \(g\circ f\) de \(A\) en \(C\)) con la función \(h\) y así
obtener \(h\circ (g\circ f)\) de \(A\) en \(D\), que es la compuesta de
las tres funciones. En este caso cabe preguntarse si es igual
\(h\circ (g\circ f)\) que \((h\circ g)\circ f\). El siguiente resultado
contesta esta pregunta.

\BeginKnitrBlock{lemma}
\protect\hypertarget{lem:unnamed-chunk-40}{}{\label{lem:unnamed-chunk-40}
}Sean \(f:A\longrightarrow B\), \(g:B\longrightarrow C\) y
\(h:\longrightarrow D\) funciones. Entonces
\(h\circ (g\circ f)=(h\circ g)\circ f\)
\EndKnitrBlock{lemma}

\BeginKnitrBlock{proof}
\iffalse{} {Demostración. } \fi{}Lo primero que debemos notar es que
tanto \(h\circ (g\circ f)\) como \((h\circ g)\circ f\) tienen el mismo
dominio y codominio. Efectivamente, \(h\circ (g\circ f)\) tiene por
dominio el conjunto \(A\), porque es dominio de \(g\circ f\) (ya vimos
antes que el dominio y el codominio de \(g\circ f\) son el dominio de
\(f\) y el codominio de \(g\) respectivamente), y su codominio es \(D\),
el codominio de \(h\). Del mismo modo \((h\circ g)\circ f\) tiene
dominio \(A\) (al ser \(dom(f)=A\)) y codominio \(D\) (que es el
codominio de \(h\circ g\)).

Ahora demostremos que para cada \(a\in A\),
\((h\circ (g\circ f))(a)=((h\circ g)\circ f)(a)\). Y es muy fácil de
ver,
\[(h\circ (g\circ f))(a)=h(g\circ f)(a)=h(g(f(a)))=(h\circ g)(f(a))=((h\circ g)\circ f)(a)\]
lo que demuestra lo que queríamos.
\EndKnitrBlock{proof}

Qué sucederá con la composición de dos funciones inyectivas, o
sobreyectivas. Esto se muestra en este resultado:

\BeginKnitrBlock{lemma}
\protect\hypertarget{lem:lema1-2}{}{\label{lem:lema1-2} }Sean
\(f:A\longrightarrow B\) y \(g:B\longrightarrow C\) dos funciones.
Entonces:

\begin{enumerate}
\def\labelenumi{\arabic{enumi})}
\item
  \(g\circ f\) es sobreyectiva si \(f\) y \(g\) lo son.
\item
  \(g\circ f\) es inyectiva si \(f\) y \(g\) lo son.

  \EndKnitrBlock{lemma}
\end{enumerate}

\BeginKnitrBlock{proof}
\iffalse{} {Demostración. } \fi{} 1) Supongamos que \(f\) y \(g\) son
funciones sobreyectivas. Sea \(c\in C\), como \(g\) es sobreyectiva,
existe \(b\in B\) tal que \(c=g(b)\). Y como \(f\) es sobreyectiva,
existe \(a\in A\) tal que \(b=f(a)\). Se tiene que dado \(c\) existe
\(a\) tal que \(c=g(b)=g(f(a))=(g\circ f)(a)\), por lo tanto
\(g\circ f\) es sobreyectiva.

\begin{enumerate}
\def\labelenumi{\arabic{enumi})}
\setcounter{enumi}{1}
\tightlist
\item
  Supongamos que \(f\) y \(g\) son funciones inyectivas. Sean
  \(a_{1}, a_{2}\in A\), tales que \(a_{1}\neq a_{2}\). Como \(f\) es
  inyectiva, se tiene que \(f(a_{1})\neq f(a_{2})\). Ahora, como
  \(f(a_{1}) \mbox{ y }f(a_{2})\in B\) y \(f(a_{1})\neq f(a_{2})\), de
  la inyectividad de \(g\) se sigue que \(g(f(a_{1}))\neq g(f(a_{2}))\),
  es decir, \((g\circ f)(a_{1})\neq (g\circ f)(a_{2})\), por lo tanto
  \(g\circ f\) es inyectiva.

  \EndKnitrBlock{proof}
\end{enumerate}

Si una función \(f\) de \(A\) en \(B\) es biyectiva, para cada
\(b\in B\) existe \(a\in A\) tal que \(f(a)=b\), y de la inyectividad se
tiene que \(a\) es único. De esta manera se puede definir una nueva
función de \(B\) en \(A\) que guarda una extrecha relación con \(f\)
(pues se define a partir de ella). Dicha función es la \textbf{inversa}
de \(f\). Definámosla formalmente.

\BeginKnitrBlock{definition}
\protect\hypertarget{def:unnamed-chunk-43}{}{\label{def:unnamed-chunk-43}
}Dada una función biyectiva \(f: A \longrightarrow B\), la
\textit{función inversa de $f$} es la función \(f^{-1}\), definida así
\(f^{-1}(b)=a\) si y solo si \(f(a)=b\).
\EndKnitrBlock{definition}

Además, para cada \(a\in A\), sea \(b=f(a)\), de donde
\((f^{-1}\circ f)(a)=f^{-1}(f(a))=f^{-1}(b)=a\). Es decir,
\(f^{-1}\circ f\) es la identidad de \(A\) (en sí mismo). Análogamente
se puede probar que \(f\circ f^{-1}\) es la identidad de \(B\). Esto es
la demostración del siguiente resultado.

\BeginKnitrBlock{lemma}
\protect\hypertarget{lem:unnamed-chunk-44}{}{\label{lem:unnamed-chunk-44}
}Dada una función \(f:A\longrightarrow B\) biyectiva, las funciones
\(f^{-1}\circ f\) y \(f\circ f^{-1}\) son iguales a la función identidad
(correspondiente a los conjuntos \(A\) y \(B\)).
\EndKnitrBlock{lemma}

Recíprocamente, si dada una función \(f: A\longrightarrow B\), existe
una función \(g:B\longrightarrow A\) tal que \(g\circ f\) y \(f\circ g\)
son la función identidad (sobre \(A\) y \(B\) respectivamente), entonces
se tiene que \(f\) es sobreyectiva, en efecto, dado \(b\in B\),
\(b=(f\circ g)(b)\), ya que \(f\circ g\) es la identidad (sobre \(B\))
por lo tanto \(b=f(g(b))=f(a)\) para algún \(a\in A\) (donde
\(a=g(b)\)). Observemos también que \(f\) es inyectiva, ya que si
\(f(a_{1})=f(a_{2})\) se tiene que \(g(f(a_{1}))=g(f(a_{2}))\), como
\(g\circ f\) es la identidad (sobre \(A\)) se tiene que \(a_{1}=a_{2}\).
Esto se puede expresar como sigue.

\BeginKnitrBlock{lemma}
\protect\hypertarget{lem:unnamed-chunk-45}{}{\label{lem:unnamed-chunk-45}
}La función \(f:A\longrightarrow B\) es biyectiva si y solo si existe
una función \(g:B\longrightarrow A\) tal que \(g\circ f\) y \(f\circ g\)
son la función identidad sobre \(A\) y \(B\) respectivamente.
\EndKnitrBlock{lemma}

\BeginKnitrBlock{definition}
\protect\hypertarget{def:unnamed-chunk-46}{}{\label{def:unnamed-chunk-46}
}Sea \(C\) un conjunto no vacío. \(\mathcal{A}(C)\) es el conjunto de
todas las funciones biyectivas de \(C\) sobre sí mismo.
\EndKnitrBlock{definition}

Respecto a este conjunto, si consideramos la operación \emph{composición
de funciones}, tenemos que \(\mathcal{A}(C)\) es cerrado bajo esta
operación, esto lo demostramos ya en el lema @ref(lem=lema1-2). Además,
como vimos antes, la composición de funciones es asociativa. Sabemos que
la identidad y la función inversa son funciones biyectivas (pertenecen
también al conjunto \(\mathcal{A}(C)\)). Es decir, tenemos un conjunto
(\(\mathcal{A}(C)\)) con una operación (la composición de funciones) que
tiene una estructura especial (la de grupo). Profundizaremos en esto en
la siguiente sección.

\section{Cardinales}\label{cardinales}

En esta sección demostraremos solo algunos resultados referidos a
cardinalidad y números cardinales, solo aquellos que nos sean realmente
útiles para el tema que nos ocupa en este trabajo. Quien desee ver las
otras demostraciones y ahondar en este tema puede referirse a .

\BeginKnitrBlock{definition}
\protect\hypertarget{def:unnamed-chunk-47}{}{\label{def:unnamed-chunk-47}
}Dos conjuntos \(A, B\) son equipotentes si existe una biyección
\(f:A \longrightarrow B\) y se denota por \(A\sim B\).
\EndKnitrBlock{definition}

\BeginKnitrBlock{theorem}
\protect\hypertarget{thm:unnamed-chunk-48}{}{\label{thm:unnamed-chunk-48}
}La equipotencia es una relación de equivalencia.
\EndKnitrBlock{theorem}

Podemos preguntarnos cuántos elementos tiene un conjunto. Una forma de
``contar'' los elementos que tiene un conjunto es la siguiente: Sean
\(A_{0}=\emptyset\) y para cada número natural \(n\), sea
\(A_{n}=\{1,2,..., n\}\). Es fácil ver que \(A_{n}=A_{m}\) si y solo si
\(n=m\) \ref{exr:ejc1}. De este modo, para ver que un conjunto \(C\)
tiene \(n\) elementos basta ver que es equipotente con \(A_{n}\), es
decir \(C\sim A_{n}\). Diremos que un conjunto es \textbf{finito} si es
equipotente con algún \(A_{n}\) para algún \(n\in\mathbb{N}\). Si un
conjunto no es finito diremos que es \textbf{infinito}.

Lo anterior nos da una idea del concepto de cardinalidad, que
formalmente se definiría como sigue.

\BeginKnitrBlock{definition}
\protect\hypertarget{def:unnamed-chunk-49}{}{\label{def:unnamed-chunk-49}
}Dado un conjunto \(C\), la clase de equivalencia definida por la
relación de equipotencia se conoce como el \textbf{cardinal} (o
\textbf{cardinalidad} o \textbf{número cardinal}) de \(C\) y se denota
por \(|C|\).
\EndKnitrBlock{definition}

En algunos libros pueden conseguirse otras definiciones de cardinalidad.
En cualquier caso esta se adecua perfectamente a los temas que
trabajaremos aquí.

\BeginKnitrBlock{remark}
\iffalse{} {Nota. } \fi{}La cardinalidad del producto cartesiano de dos
conjuntos \(A, B\) es el producto \(|A||B|\).
\EndKnitrBlock{remark}

\subsection{Ejercicios}\label{ejercicios}

\BeginKnitrBlock{exercise}
\protect\hypertarget{exr:ejc1}{}{\label{exr:ejc1} }Dados los conjuntos
\(A_{0}=\emptyset\) y para cada número natural \(n\), sea
\(A_{n}=\{1,2,..., n\}\). Demuestre que \(A_{n}=A_{m}\) si y solo si
\(n=m\).
\EndKnitrBlock{exercise}

\section{Teoría de Grupos}\label{teoria-de-grupos}

En esta sección estudiaremos un objeto matemático de gran importancia,
los grupos. En la sección anterior vimos un grupo que surgió de manera
natural, \(\mathcal{C}\) junto a la operación composición de funciones,
pero el grupo más familiar es el de los números enteros (con la
operación suma), con el que nos topamos desde la infancia; en ambos
ejemplos vemos que la operación es asociativa, tiene un elemento neutro
(la función identidad en el primer caso y el número cero en el caso de
los números naturales) y un elemento inverso (la función inversa en un
caso y el opuesto en el caso de los naturales).

A continuación presentaremos la definición formal de grupo así como un
amplio repertorio de resultados bien conocidos en álgebra respecto a
este objeto.

\smallskip

Dado un conjunto no vacío \(G\), una \emph{operación binaria} es una
función \(G\times G\longrightarrow G\). Comunmente se usan las
notaciones \(a\ast b\) o \(a\cdot b\) para denotar la imagen de
\((a,b)\) por la función, aunque puede también usarse \(ab\) (obviando
el punto como se hace para expresar el producto de dos números) o
incluso \(a+b\) cuando la operación es la suma usual (como sucede con
los números enteros).

\BeginKnitrBlock{definition}
\protect\hypertarget{def:unnamed-chunk-51}{}{\label{def:unnamed-chunk-51}
}Un par \((G,\ast)\), donde \(G\) es un conjunto no vacío y una operacón
binaria \(\ast:G\times G\longrightarrow G\), forman un \textbf{grupo}
si:

\begin{enumerate}
\def\labelenumi{\roman{enumi})}
\item
  Para todo \(a,b,c\in G\), \((a\ast b)\ast c=a\ast (b\ast c)\). Es
  decir, la operación es \emph{asociativa}.
\item
  Existe un elemento \(e\in G\) tal que \(a\ast e=e\ast a=a\).
  Llamaremos a tal elemento \emph{neutro (o identidad) bilateral} de
  \(G\).
\item
  Para todo \(a\in G\), existe un elemento \(a^{-1}\in G\) tal que
  \(a^{-1}\ast a=a\ast a^{-1}=e\), llamado \emph{inverso} de \(a\).

  \EndKnitrBlock{definition}
\end{enumerate}

\BeginKnitrBlock{remark}
\iffalse{} {Nota. } \fi{}Puede hacerse referencia al grupo nombrando
solo el conjunto \(G\) cuando quede claro cual es la operación.

Si una operacón binaria \(\ast:G\times G\longrightarrow G\) es
asociativa (i.), se dice que \((G,\ast)\) es un \emph{semigrupo}. Un
\textbf{monoide} es un semigrupo con identidad (ii.). De este modo, se
puede decir que un grupo es un monoide con inverso (bilateral).
\EndKnitrBlock{remark}

Un semigrupo \(G\) se llamará \textbf{abeliano} o \textbf{commutativo}
si la operación es

\begin{enumerate}
\def\labelenumi{\roman{enumi})}
\setcounter{enumi}{3}
\tightlist
\item
  Commutativa, es decir, \(a*b=b*a\), para todo \(a, b\in G\).
\end{enumerate}

El \textbf{orden} de un grupo \(G\) es la cantidad de elementos que
tiene el grupo, es decir, la cardinalidad de \(G\) (\(|G|\)). También se
denota \(o(G)\) Decimos que un grupo es de \textbf{orden finito} (o
simplemente finito) si \(|G|\) es finito. En caso contrario decimos que
el grupo es \textbf{infinito}.

\BeginKnitrBlock{example}
\protect\hypertarget{exm:unnamed-chunk-53}{}{\label{exm:unnamed-chunk-53}
}Como ya lo hemos mencionado, el conjunto de los números enteros,
\(\mathbb{Z}\) con la operación suma (la suma usual de enteros), forman
un grupo. El lector podrá verificar fácilmente que la operación suma es
cerrada, es asociativa, que el cero es el elemento neutro (\(e\) en la
definición) y que cada elemento tiene un inverso (\(a^{-1}=-a\)). Además
es claro que se trata de un grupo abeliano (la suma es una operación
commutativa).
\EndKnitrBlock{example}

\BeginKnitrBlock{example}
\protect\hypertarget{exm:unnamed-chunk-54}{}{\label{exm:unnamed-chunk-54}
}Dado el conjunto \(G={1,-1}\). Definimos la operación
\(\ast:G\times G \longrightarrow G\) como el producto de números reales
usual. El par \((G,\ast)\) forman un grupo abeliano de orden \(2\).
\EndKnitrBlock{example}

\BeginKnitrBlock{example}
\protect\hypertarget{exm:unnamed-chunk-55}{}{\label{exm:unnamed-chunk-55}
}El conjunto de los números racionales \(\mathbb{Q}\) con la suma usual,
es un grupo abeliano.
\EndKnitrBlock{example}

\BeginKnitrBlock{example}
\protect\hypertarget{exm:unnamed-chunk-56}{}{\label{exm:unnamed-chunk-56}
}Consideremos un cuadrado cuyos vértices estan numerados
consecutivamente \(1,2,3,4\) centrado en el origen del palno cartesiano
y de lados paralelos a los ejes coordenados.

Sea \(C_{4}\) el conjunto formado por las siguientes transformaciones:
\(R\), una rotación de \(90º\) del cuadrado. \(R^{2}\) una rotación de
\(180º\) del cuadrados. \(R^{3}\) una rotación de \(270º\) del cuadrado
(todas en el sentido de las agujas del reloj, centradas en el origen).
\(I\), una rotación de \(360º\) (igual que antes en sentido horario,
centrada en el origen). \(T_{x}\) y \(T_{y}\), reflexiones sobre los
ejes \(x\) y \(y\) respectivamente y \(T_{I}\) y \(T_{II}\) reflexiones
sobre las diagonales que pasan por los vértices que están en el primer y
tercer cuadrante (la primera) y en el segundo y cuarto cuadrante (la
segunda). Con la operación \emph{composición de funciones}, el conjunto
\(C_{4} = {R, R^{2}, R^{3}, I, T_{x}, T_{y}, T_{I}, T_{II}}\) es un
grupo no abeliano de orden \(8\) llamado el \textbf{grupo de simetría
del cuadrado}.
\EndKnitrBlock{example}

\BeginKnitrBlock{example}
\protect\hypertarget{exm:unnamed-chunk-57}{}{\label{exm:unnamed-chunk-57}
}Sea \(C\) un conjunto no vacío y \(\mathcal{A}(C)\) el conjunto de
todas las biyecciones de \(C\) en si mismo. Con la operación composición
de funciones vista en la sección anterior, \(\mathcal{A}(C)\) forma un
grupo (no abeliano). En efecto, la composición de funciones biyectivas
es asociativa, la identidad es una función biyectiva y toda biyección
tiene una inversa. Los elementos de \(\mathcal{A}(C)\) son llamados
\textbf{permutaciones} y \(\mathcal{A}(C)\) es llamado el grupo de
permutaciones sobre \(C\). Si \(C=\{1,2, ..., n\}\), entonces
\(\mathcal{A}(C)\) es llamdo el \textbf{grupo simétrico sobre \(n\)
letras} y se denota \(S_{n}\). Se puede ver que \(|S_{n}|=n!\)
(ejercicio \ref{exr:ejc3}).
\EndKnitrBlock{example}

\BeginKnitrBlock{example}
\protect\hypertarget{exm:ejm1-21}{}{\label{exm:ejm1-21} }Dados \(G\) y \(H\)
dos grupos con identidades \(e_{G}\) y \(e_{H}\) respectivamente.
Consideremos el producto cartesiano \(G\times H\) y la operación binaria
\((a,b)\ast (c,d)=(a\ast c,b\ast d)\) donde \(a\ast c\in G\) y
\(b\ast d\in H\). Con esta operación \(G\times H\) es un grupo con
identidad \((e_{G}, e_{H})\) y con inverso \((a^{-1}, b^{-1})\) para
cada elemento \((a,b)\in G\times H\).
\EndKnitrBlock{example}

\subsection{Ejercicios}\label{ejercicios-1}

\BeginKnitrBlock{exercise}
\protect\hypertarget{exr:unnamed-chunk-58}{}{\label{exr:unnamed-chunk-58}
}Sea \(G\) un grupo y \(C\) un conjunto no vacío. Sea \(M(C,G)\) el
conjunto de todas las funciones \(f:C\longrightarrow G\). Definamos la
operación de grupo como la suma de funciones, es decir, para cada
\(f,g\in M(C,G)\), \(f\ast g = f +g\). Demuestre que \(M(C,G)\) es un
grupo, es abeliano si \(G\) lo es.
\EndKnitrBlock{exercise}

\BeginKnitrBlock{exercise}
\protect\hypertarget{exr:ejc3}{}{\label{exr:ejc3} }Demuestre que el grupo
simétrico sobre \(n\) letras es de orden \(n!\).
\EndKnitrBlock{exercise}

\BeginKnitrBlock{exercise}
\protect\hypertarget{exr:unnamed-chunk-59}{}{\label{exr:unnamed-chunk-59}
}Demuestre que el grupo del ejemplo @ref(exm: ejm1-21) es un grupo de
orden \(|G||H|\). Además muestre que \(G\times H\) es un grupo abeliano
si \(G\) y \(H\) lo son.
\EndKnitrBlock{exercise}

\chapter{Vectores}\label{vectores}

\chapter{Espacios vectoriales}\label{espacios-vectoriales}

\chapter{Matrices}\label{matrices}

\chapter{Autovalores y autovectores}\label{autovalores-y-autovectores}

\chapter{Cálculo en vectores y
matrices}\label{calculo-en-vectores-y-matrices}

\chapter{Transformaciones lineales}\label{transformaciones-lineales}

\chapter{Producto escalar}\label{producto-escalar}

\cleardoublepage 

\appendix \addcontentsline{toc}{chapter}{\appendixname}


\chapter{Software Tools}\label{software-tools}

For those who are not familiar with software packages required for using
R Markdown, we give a brief introduction to the installation and
maintenance of these packages.

\section{R and R packages}\label{r-and-r-packages}

R can be downloaded and installed from any CRAN (the Comprehensive R
Archive Network) mirrors, e.g., \url{https://cran.rstudio.com}. Please
note that there will be a few new releases of R every year, and you may
want to upgrade R occasionally.

To install the \textbf{bookdown} package, you can type this in R:

\begin{Shaded}
\begin{Highlighting}[]
\KeywordTok{install.packages}\NormalTok{(}\StringTok{"bookdown"}\NormalTok{)}
\end{Highlighting}
\end{Shaded}

This installs all required R packages. You can also choose to install
all optional packages as well, if you do not care too much about whether
these packages will actually be used to compile your book (such as
\textbf{htmlwidgets}):

\begin{Shaded}
\begin{Highlighting}[]
\KeywordTok{install.packages}\NormalTok{(}\StringTok{"bookdown"}\NormalTok{, }\DataTypeTok{dependencies =} \OtherTok{TRUE}\NormalTok{)}
\end{Highlighting}
\end{Shaded}

If you want to test the development version of \textbf{bookdown} on
GitHub, you need to install \textbf{devtools} first:

\begin{Shaded}
\begin{Highlighting}[]
\ControlFlowTok{if}\NormalTok{ (}\OperatorTok{!}\KeywordTok{requireNamespace}\NormalTok{(}\StringTok{'devtools'}\NormalTok{)) }\KeywordTok{install.packages}\NormalTok{(}\StringTok{'devtools'}\NormalTok{)}
\NormalTok{devtools}\OperatorTok{::}\KeywordTok{install_github}\NormalTok{(}\StringTok{'rstudio/bookdown'}\NormalTok{)}
\end{Highlighting}
\end{Shaded}

R packages are also often constantly updated on CRAN or GitHub, so you
may want to update them once in a while:

\begin{Shaded}
\begin{Highlighting}[]
\KeywordTok{update.packages}\NormalTok{(}\DataTypeTok{ask =} \OtherTok{FALSE}\NormalTok{)}
\end{Highlighting}
\end{Shaded}

Although it is not required, the RStudio IDE can make a lot of things
much easier when you work on R-related projects. The RStudio IDE can be
downloaded from \url{https://www.rstudio.com}.

\section{Pandoc}\label{pandoc}

An R Markdown document (\texttt{*.Rmd}) is first compiled to Markdown
(\texttt{*.md}) through the \textbf{knitr} package, and then Markdown is
compiled to other output formats (such as LaTeX or HTML) through
Pandoc.\index{Pandoc} This process is automated by the
\textbf{rmarkdown} package. You do not need to install \textbf{knitr} or
\textbf{rmarkdown} separately, because they are the required packages of
\textbf{bookdown} and will be automatically installed when you install
\textbf{bookdown}. However, Pandoc is not an R package, so it will not
be automatically installed when you install \textbf{bookdown}. You can
follow the installation instructions on the Pandoc homepage
(\url{http://pandoc.org}) to install Pandoc, but if you use the RStudio
IDE, you do not really need to install Pandoc separately, because
RStudio includes a copy of Pandoc. The Pandoc version number can be
obtained via:

\begin{Shaded}
\begin{Highlighting}[]
\NormalTok{rmarkdown}\OperatorTok{::}\KeywordTok{pandoc_version}\NormalTok{()}
\NormalTok{## [1] '1.19.2.1'}
\end{Highlighting}
\end{Shaded}

If you find this version too low and there are Pandoc features only in a
later version, you can install the later version of Pandoc, and
\textbf{rmarkdown} will call the newer version instead of its built-in
version.

\section{LaTeX}\label{latex}

LaTeX\index{LaTeX} is required only if you want to convert your book to
PDF. The typical choice of the LaTeX distribution depends on your
operating system. Windows users may consider MiKTeX
(\url{http://miktex.org}), Mac OS X users can install MacTeX
(\url{http://www.tug.org/mactex/}), and Linux users can install TeXLive
(\url{http://www.tug.org/texlive}). See
\url{https://www.latex-project.org/get/} for more information about
LaTeX and its installation.

Most LaTeX distributions provide a minimal/basic package and a full
package. You can install the basic package if you have limited disk
space and know how to install LaTeX packages later. The full package is
often significantly larger in size, since it contains all LaTeX
packages, and you are unlikely to run into the problem of missing
packages in LaTeX.

LaTeX error messages may be obscure to beginners, but you may find
solutions by searching for the error message online (you have good
chances of ending up on
\href{http://tex.stackexchange.com}{StackExchange}). In fact, the LaTeX
code converted from R Markdown should be safe enough and you should not
frequently run into LaTeX problems unless you introduced raw LaTeX
content in your Rmd documents. The most common LaTeX problem should be
missing LaTeX packages, and the error may look like this:

\begin{Shaded}
\begin{Highlighting}[]
\NormalTok{! LaTeX Error: File `titling.sty' not found.}

\NormalTok{Type X to quit or <RETURN> to proceed,}
\NormalTok{or enter new name. (Default extension: sty)}

\NormalTok{Enter file name: }
\NormalTok{! Emergency stop.}
\NormalTok{<read *> }
         
\NormalTok{l.107 ^^M}

\NormalTok{pandoc: Error producing PDF}
\NormalTok{Error: pandoc document conversion failed with error 43}
\NormalTok{Execution halted}
\end{Highlighting}
\end{Shaded}

This means you used a package that contains \texttt{titling.sty}, but it
was not installed. LaTeX package names are often the same as the
\texttt{*.sty} filenames, so in this case, you can try to install the
\texttt{titling} package. Both MiKTeX and MacTeX provide a graphical
user interface to manage packages. You can find the MiKTeX package
manager from the start menu, and MacTeX's package manager from the
application ``TeX Live Utility''. Type the name of the package, or the
filename to search for the package and install it. TeXLive may be a
little trickier: if you use the pre-built TeXLive packages of your Linux
distribution, you need to search in the package repository and your
keywords may match other non-LaTeX packages. Personally, I find it
frustrating to use the pre-built collections of packages on Linux, and
much easier to install TeXLive from source, in which case you can manage
packages using the \texttt{tlmgr} command. For example, you can search
for \texttt{titling.sty} from the TeXLive package repository:

\begin{Shaded}
\begin{Highlighting}[]
\ExtensionTok{tlmgr}\NormalTok{ search --global --file titling.sty}
\CommentTok{# titling:}
\CommentTok{#    texmf-dist/tex/latex/titling/titling.sty}
\end{Highlighting}
\end{Shaded}

Once you have figured out the package name, you can install it by:

\begin{Shaded}
\begin{Highlighting}[]
\ExtensionTok{tlmgr}\NormalTok{ install titling  # may require sudo}
\end{Highlighting}
\end{Shaded}

LaTeX distributions and packages are also updated from time to time, and
you may consider updating them especially when you run into LaTeX
problems. You can find out the version of your LaTeX distribution by:

\begin{Shaded}
\begin{Highlighting}[]
\KeywordTok{system}\NormalTok{(}\StringTok{'pdflatex --version'}\NormalTok{)}
\NormalTok{## pdfTeX 3.14159265-2.6-1.40.18 (TeX Live 2017)}
\NormalTok{## kpathsea version 6.2.3}
\NormalTok{## Copyright 2017 Han The Thanh (pdfTeX) et al.}
\NormalTok{## There is NO warranty.  Redistribution of this software is}
\NormalTok{## covered by the terms of both the pdfTeX copyright and}
\NormalTok{## the Lesser GNU General Public License.}
\NormalTok{## For more information about these matters, see the file}
\NormalTok{## named COPYING and the pdfTeX source.}
\NormalTok{## Primary author of pdfTeX: Han The Thanh (pdfTeX) et al.}
\NormalTok{## Compiled with libpng 1.6.29; using libpng 1.6.29}
\NormalTok{## Compiled with zlib 1.2.11; using zlib 1.2.11}
\NormalTok{## Compiled with xpdf version 3.04}
\end{Highlighting}
\end{Shaded}

\bibliography{book.bib,packages.bib}

\backmatter
\printindex

\end{document}
